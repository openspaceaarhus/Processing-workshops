\documentclass{beamer}

% \mode<presentation>
\usetheme{Luebeck} %Warsaw
% \usecolortheme{seahorse}
% \usecolortheme{lily}
\usecolortheme[RGB={51,255,0}]{structure}

\usepackage[utf8]{inputenc}
\usepackage[danish]{babel}
\usepackage{graphicx}
\usepackage{url}
\usepackage[normalem]{ulem}

\graphicspath{{images/}}
\newcommand{\FIG}[2]{
  \begin{figure}[]
    \centering
    \includegraphics[width=0.95\textwidth,keepaspectratio]{#1}
    \caption{#2}
    \label{fig:#1}
  \end{figure}
}

\newcommand{\FIGMED}[2]{
  \begin{figure}[]
    \centering
    \includegraphics[width=0.75\textwidth,keepaspectratio]{#1}
    \caption{#2}
    \label{fig:#1}
  \end{figure}
}


\newcommand{\FIGSMALL}[2]{
  \begin{figure}[htbp]
    \centering
    \includegraphics[width=0.4\textwidth,keepaspectratio]{#1}
    \caption{#2}
    \label{fig:#1}
  \end{figure}
}


\title{Processing.org workshops\\Workshop 4}
\author{Open Space Aarhus}
\date{\today}
\institute[Bryggervej 30]{Bryggervej 30, 8240 Århus N}

% logo
\pgfdeclareimage[height=1.3cm]{university-logo}{osaa_logo_neon_rgb}
\logo{\pgfuseimage{university-logo}}

\begin{document}

\begin{frame}[label=titlepage]
  \titlepage
\end{frame}

\begin{frame}
  \frametitle{Dagens program}
  \begin{columns}
    \begin{column}{0.5\textwidth}
  
  \begin{itemize}
  \item Introduktion
  \item Resume af sidste gang
  \item Kode
    \begin{itemize}
    \item Klasser
    \end{itemize}

  \item Grafik
    \begin{itemize}
    \item Billeder
    \item Fonte (Tekst)
    \item Snevejr
    \item Pixel-pushing - måske
    \end{itemize}

  \item \emph{Afslutning}
  \end{itemize}						

    \end{column}
    
%    \begin{column}{0.5\textwidth}
%      \FIG{dagensprogram}{Workshop 3}
%    \end{column}
  \end{columns}
    
\end{frame}


\begin{frame}
  \frametitle{Introduktion}
  
  \begin{block}{Slides og processing filer}
    \url{http://poodle/processing}   
  \end{block}
  {\tiny Slides kan sikkert bruges til at kigge i eller kopiere fra.}
\end{frame}


\begin{frame}
  \frametitle{OpenProcessing.org}
  \begin{block}{Hvad har de flittige lavet}
    \begin{itemize}
    \item \url{http://www.openprocessing.org/classrooms/?classroomID=1075}
    \end{itemize}
  \end{block}
\end{frame}


\begin{frame}[fragile]
  \frametitle{Resume af sidste gang}
  
  \begin{itemize}
  \item Kode
    \begin{itemize}
    \item Arrays:
      \begin{itemize}
      \item \texttt{float[] boldX = new float[10];}\\
      \item \texttt{boldX[5] = 200;}\\
      \item \texttt{ellipse(boldX[5], boldY[5], 5, 5);}\\
      \end{itemize}
    \item Løkker:
      \begin{itemize}
      \item \texttt{while (betingelse) \{ ... \}}\\
      \item \texttt{for (start; betingelse; opdatering;) \{ ... \}}\\
      \end{itemize}
    \end{itemize}
  \item Grafik
    \begin{itemize}
    \item Mange bolde
    \item Fyrværkeri
    \end{itemize}  
  \item Spørgsmål?
  \end{itemize}  
\end{frame}


\begin{frame}[fragile]
  \frametitle{Det er jo jul!}
  
  \begin{itemize}
  \item Et animeret julekort
  \item Snevejr for at fortsætte partikel temaet
  \item Brug klasser til at strukturere koden
  \end{itemize}
  
\end{frame}


\begin{frame}[fragile]
  \frametitle{Snefnug}
  
  \begin{columns}
  \begin{column}{0.5\textwidth}
\begin{verbatim}
float posX = 200;
float posY = 200;
float velX = 0;
float velY = 1;
\end{verbatim}
  \end{column}
  \begin{column}{0.5\textwidth}
\begin{verbatim}
  posX += velX;
  posY += velY;
  
  if (posY > height) {
    posY = 0;
  }
\end{verbatim}  
  \end{column}
  \end{columns}

\end{frame}


\begin{frame}[fragile]
  \frametitle{Funktioner - fra sidste gang}
    
\begin{verbatim}
returtype navn(parametre) {
 // implementation
}
\end{verbatim}
  
  \begin{block}{Eksempler}
  {\tiny
\begin{verbatim}
void draw() {
}

void drawTarget(float x, float y) {
  fill(255, 0, 0);
  ellipse(x, y, 40, 40);
  fill(255, 255, 255);  
  ellipse(x, y, 30, 30);
  fill(255, 0, 0);
  ellipse(x, y, 10, 10);  
}

float vinkelTilMus(float x, float y) {
  return atan2(mouseY - y, mouseX - x);
}

\end{verbatim}    
}
  \end{block}

\end{frame}


\begin{frame}[fragile]
  \frametitle{Snefnug med funktioner}
 
  \begin{columns}
  \begin{column}{0.5\textwidth}
\begin{verbatim}
void update() {
  posX += velX;
  posY += velY;
  
  if (posY > height) {
    posY = 0;
  }
}

void paint() {
  ellipse(posX, posY, 10, 10);  
}
\end{verbatim}
  \end{column}
  \begin{column}{0.5\textwidth}
\begin{verbatim}
void draw() {
  background(50, 100, 200);
  update();
  paint();
}\end{verbatim}  
  \end{column}
  \end{columns}

\end{frame}


\begin{frame}
  \frametitle{Klasser}

  \begin{itemize}
  \item Brugerdefinerede datatyper
  \item Samler variable og de funktioner der benytter variablene i en logisk enhed
  \item Genbrugelig
  \item Masser af nyttige klasser på nettet
  \item Man behøver ikke forstå implementationen for at bruge dem! 
  \item Abstraktion
  \item Objektorienteret programmering
  \end{itemize}

\end{frame}

\begin{frame}
  \frametitle{Klasser og Objekter}

  \begin{block}{Klasser}
  Abstrakt beskrivelse so alle objekter af typen overholder 
  Definerer hvilke variable og funktioner der er tilgængelige.
  
  En brugerdefineret datatype.
  \end{block}
  
  \begin{block}{Objekter}
  Faktiske instanser af klassen. De "virkelige" objekter som
  man arbejder med.
  
  En variabel af typen.
  \end{block}
  
\end{frame}


\begin{frame}[fragile]
  \frametitle{Din første klasse}

  \begin{columns}
  \begin{column}{0.5\textwidth}
{\small
\begin{verbatim}
class Fnug {
  float posX = 200;
  float posY = 200;
  float velX = 0;
  float velY = 1;

  void update() {
    posX += velX;
    posY += velY;
    if (posY > height) {
      posY = 0;
    }
  }
  
  void paint() {
    ellipse(posX, posY, 10, 10);  
  }
}
\end{verbatim}
}
  \end{column}
  
  \begin{column}{0.5\textwidth}
{\small
\begin{verbatim}

// variabel af typen Fnug
// initialiseret til et nyt Fnug objekt
Fnug fnug = new Fnug();

void draw() {
  background(50, 100, 200);
  fnug.update();
  fnug.paint();
}
\end{verbatim}  
}
  \end{column}
  \end{columns}

\end{frame}


\begin{frame}[fragile]
  \frametitle{Tabs}
  
  \begin{itemize}
  \item{Brug tabs til at organisere koden}
  \item{God ide: En tab per klasse}
  \item{Giv tab'en samme navn som klassen}
  \end{itemize}
  
\end{frame}


\begin{frame}[fragile]
  \frametitle{Arrays - fra sidste gang}
  
  Et array er en opslagstabel. Man kan lave et array med et fast antal pladser.
  Derefter kan man skrive og læse værdier på de enkelte pladser i arrayet.
  
  \begin{block}{Syntax}
  \begin{itemize}
  \item erklæring: type[] navn;
  \item initialisering: float[] boldX = new float[10];
  \item tildeling: boldX[5] = 100; 
  \item læsning: ellipse(boldX[5], boldY[5], 30, 30);
  \end{itemize}
  \end{block}
  
  Bemærk at det første element i et array har indeks 0. Det vil fx sige at et array med 10 elementer indekseres med tallene 0-9.
\end{frame}


\begin{frame}[fragile]
  \frametitle{Arrays af objekter}
  
  \begin{itemize}
  \item Erklæring:  {\texttt Fnug[] snevejr;}
  \item Initialisering:  {\texttt Fnug[] snevejr = new Fnug[10];}
  \item Tildeling:  {\texttt snevejr[5] = new Fnug();} VIGTIGT!!!
  \item Metodekald:  {\texttt snevejr[5].update(); }  
  \item Læsning:  {\texttt Fnug fnug = snevejr[5];}
  \item Metodekald:  {\texttt fnug.update(); }
  \end{itemize}
  
\end{frame}


\begin{frame}[fragile]
  \frametitle{Skrifttyper}
  \begin{itemize}
  \item Klassen PFont holder en skrifttype
  \item Brug Tools/Create Font
  \item Filen ender i en data folder
  \item Variabel der kan holde en font
    \begin{itemize}
    \item \texttt{PFont font;}
    \end{itemize}
  \item Indlæs font
    \begin{itemize}
    \item \texttt{font = loadFont("Verdana-Bold-48.vlw");}
    \end{itemize}
  \item Vælg font
    \begin{itemize}
    \item \texttt{textFont(font, 100);}
    \end{itemize}
  \item Tegn tekst
    \begin{itemize}
    \item \texttt{text("God Jul", 100, 100);}
    \end{itemize}
  \end{itemize}
\end{frame}



\begin{frame}[fragile]
  \frametitle{Billeder - Indlæs og tegn}
  \begin{itemize}
  \item Klassen PImage repræsenterer et billede
  \item Variabel der kan holde et billede
    \begin{itemize}
    \item \texttt{PImage img;}
    \end{itemize}
  \item Indlæs billede
    \begin{itemize}
    \item \texttt{img = loadImage("osaa.png");}
    \end{itemize}
  \item Tegn billede
    \begin{itemize}
    \item \texttt{image(img, x, y, width, height);}
    \end{itemize}
  \end{itemize}
\end{frame}


\begin{frame}[fragile]
  \frametitle{Pixels - teori}
  \begin{itemize}
  \item Direkte adgang til pixel data
  \item Een kontinuerlig blok i hukommelsen
  \item Skal selv udregne indeks fra x,y koordinater
  \end{itemize}
\end{frame}

\begin{frame}[fragile]
  \frametitle{Pixels - læs og skriv}
  \begin{itemize}
  \item readPixels()
  \item pixels[x + y * width] = color(255);
  \item color c = pixels[x + y * width];
  \item updatePixels()
  \end{itemize}
\end{frame}

\begin{frame}[fragile]
  \frametitle{Snevejr}
  \begin{itemize}
  \item Brug Fnug klassen til at simulere svævende fnug
  \item Brug \texttt{pixels} arrayet til at sætte fastlåste fnug
  \item Test for kollision med pixels
  \item Udfordring: Undgå tårne og pyramider
  \end{itemize}
\end{frame}


\begin{frame}
  \frametitle{Libraries / Biblioteker}
  \begin{itemize}
  \item Eksempler på indbyggede
    \begin{itemize}
    \item PFont - skrifttyper
    \item PImage - indlæs bitmap-baserede billeder
    \item PVector - en matematisk vektor
    \item OpenGL - 3D grafik og hardware acceleration
    \item Minim - Lyd input og output
    \end{itemize}
  \item Eksempler på udvidelser
    \begin{itemize}
    \item controlP5 - GUI
    \end{itemize}
  \end{itemize}
\end{frame}

\begin{frame}
  \frametitle{Tak for denne gang}

  \begin{itemize}
  \item Hvad syntes \emph{du} om workshoppen?
  \item Hvad så nu?
  \end{itemize}

  \begin{block}{Klasseværelset}
    \url{www.openprocessing.org/classrooms/?classroomID=1075}
  \end{block}

\end{frame}


\begin{frame}
  \frametitle{Resume}
  \begin{columns}
    \begin{column}{0.5\textwidth}
      \begin{itemize}
      \item Variable
      \item Primitive datatyper
      \item Operatorer
      \item Forgreninger
      \item Løkker
      \item Funktioner
      \item Arrays
      \item Klasser
      \end{itemize}
    \end{column}
    \begin{column}{0.5\textwidth}        
      \begin{itemize}
      \item Koordinatsystemet
      \item Farver
      \item Tegnefunktioner
      \item Input fra tastatur og mus
      \item Simpel partikel simulering
      \item Billeder
      \item Pixeldata
      \end{itemize}
    \end{column}
  \end{columns}    
\end{frame}


\begin{frame}
  \frametitle{Hvad så nu}
      \begin{itemize}
      \item På egen hånd
      \item Kontaktsport!
      \item Kom og spørg
      \item Næste år
        \begin{itemize}
        \item Tema-kode-aftener
        \end{itemize}
      \end{itemize}
\end{frame}

\end{document}


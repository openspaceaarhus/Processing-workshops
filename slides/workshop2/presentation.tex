
\documentclass{beamer}

% \mode<presentation>
\usetheme{Luebeck} %Warsaw
% \usecolortheme{seahorse}
% \usecolortheme{lily}
\usecolortheme[RGB={51,255,0}
]{structure}

\usepackage[utf8]{inputenc}
\usepackage[danish]{babel}
\usepackage{graphicx}
\usepackage{url}
\usepackage[normalem]{ulem}

\graphicspath{{images/}}
\newcommand{\FIG}[2]{
  \begin{figure}[]
    \centering
    \includegraphics[width=0.95\textwidth,keepaspectratio]{#1}
    \caption{#2}
    \label{fig:#1}
  \end{figure}
}

\newcommand{\FIGMED}[2]{
  \begin{figure}[]
    \centering
    \includegraphics[width=0.75\textwidth,keepaspectratio]{#1}
    \caption{#2}
    \label{fig:#1}
  \end{figure}
}


\newcommand{\FIGSMALL}[2]{
  \begin{figure}[htbp]
    \centering
    \includegraphics[width=0.4\textwidth,keepaspectratio]{#1}
    \caption{#2}
    \label{fig:#1}
  \end{figure}
}


\title{Processing.org workshops\\Intro}
\author{Open Space Aarhus}
\date{\today}
\institute[Bryggervej 30]{Bryggervej 30, 8240 Århus N}

% logo
\pgfdeclareimage[height=1.3cm]{university-logo}{osaa_logo_neon_rgb}
\logo{\pgfuseimage{university-logo}}

\begin{document}

\begin{frame}[label=titlepage]
  \titlepage
\end{frame}

\begin{frame}
  \frametitle{Dagens program}
  \begin{itemize}
  \item \url{http://openprocessing.org}
  \item Hvad er et partikelsystem
  \item Opsumering variable
  \item Kode
    \begin{itemize}
    \item Forgreninger
    \item Løkker
    \item Klasser
    \end{itemize}

  \item \emph{Afslutning}
    
  \end{itemize}						
\end{frame}


\begin{frame}
  \frametitle{OpenProcessing.org}
  \begin{block}{Hvad har de flittige lavet}
    \begin{itemize}
    \item \url{http://www.openprocessing.org/classrooms/?classroomID=1075}
    \end{itemize}
  \end{block}
\end{frame}



\begin{frame}
  \frametitle{Partikelsystem}

  \url{http://vimeo.com/28256186}  

  \url{https://www.youtube.com/watch?v=ncykt-YJO1M}
\end{frame}


\begin{frame}
  \frametitle{Introduktion}
  
  Først tager vi lige en hurtig og tildels teoretisk gennemgang af
  centrale begreber fra sidst + det nye vi skal bruge i dag. \\
  \vspace{2cm}
  Derefter skal vi har \emph{beskidte fingre}.

  \begin{block}{Slides og processing filer}
    \url{http://poodle/processing}   
  \end{block}

\end{frame}


\begin{frame}
  \frametitle{Variable - \emph{erklæringer}}
  
  Der er indbyggede variable som
  \begin{itemize}
  \item mouseX
  \item mouseY
  \item width
  \end{itemize}

  Du kan \emph{erklære} din egne variable
  \begin{itemize}
  \item int x;
  \item float y;
  \end{itemize}

  \begin{block}{Erklæring}
    datatype navn;
  \end{block}

\end{frame}

\begin{frame}
  \frametitle{Variable - \emph{tildelinger}}
  
  Du kan \emph{tildele} en værdi til variable.
  \begin{itemize}
  \item x = 42;
  \item y = 3.14;
  \end{itemize}


  \begin{block}{OBS datatyper}
    int x; \\
    x = 3.14 \\
    x er nu 3, fordi den automagisk laver den om til et heltal. \\
    \sout{x = ``noget tekst''} \textbf{Boom}
  \end{block}

\end{frame}



\begin{frame}[fragile]
  \frametitle{Datatyper}
  \begin{columns}
    \begin{column}{0.4\textwidth}
      \begin{block}{int}
        Heltal (integer) : $1,2,42$
      \end{block}
      \begin{block}{float}
        komma tal: $3.14 \times 10^{42}$
      \end{block}

    \end{column}
    \begin{column}{0.6\textwidth}    
\begin{verbatim}
int x = 42;

float y = 3.14;
\end{verbatim}

    \end{column}
  \end{columns}  
\end{frame}

\begin{frame}[fragile]
  \frametitle{Datatyper - lidt andre}
  \begin{columns}
    \begin{column}{0.4\textwidth}
      \begin{block}{double}
        ligsom \texttt{float}, bare flere decimaler
      \end{block}
      \begin{block}{char}
        En \emph{byte(0-255)} - kan gemme eet bogstav.
      \end{block}
      \begin{block}{String}
        Tekst stykker: ``I'm a string''
      \end{block}
    \end{column}
    \begin{column}{0.6\textwidth}    
\begin{verbatim}

double z = 8.92838429338;

//OBS  ' og ikke " 
char c = 'X'; 

String hello = "world";

\end{verbatim}

    \end{column}
  \end{columns}  
\end{frame}



\begin{frame}[fragile]
  \frametitle{Boolean - \emph{endnu en datatype}}
  \begin{columns}
    \begin{column}{0.4\textwidth}
      \begin{block}{Boolean}
        Kan være \emph{sand} eller \emph{falsk}
      \end{block}
    \end{column}
    \begin{column}{0.6\textwidth}    
      \texttt{\emph{boolean} nemt = true; }\\
      \texttt{\emph{boolean} justinRocks = false; }
    \end{column}
  \end{columns}  
\end{frame}


\begin{frame}[fragile]
  \frametitle{Sammenligninger}
  \begin{columns}
    \begin{column}{0.5\textwidth}
      \begin{block}{Operatorer}
        \begin{table}[h]
          \centering
          \begin{tabular}{lr}
            \hline
            Lighed & $==$ \\
            Ikke ens & $!=$\\
            Større end & $> $\\
            Mindre end & $<$\\
            Større end eller lig & $>=$\\
            Ditto for mindre & $<=$\\
            
          \end{tabular}
        \end{table}
      \end{block}
    \end{column}
    \begin{column}{0.5\textwidth}
\begin{verbatim} 
int x = 42;
int y = 0;

boolean foo = (x == y);
//false

boolean bar = (x >= y);
//true

\end{verbatim}
    \end{column}
  \end{columns}
\end{frame}


\begin{frame}[fragile]
  \frametitle{Boolske udtryke}
  \begin{columns}
    \begin{column}{0.5\textwidth}
      \begin{block}{Operatorer}
        \begin{table}[h]
          \centering
          \begin{tabular}{llr}
            Dansk & teknisk & kode \\
            \hline
            og & \texttt{and} & $\&\&$ \\
            enten & \texttt{or} & $|| $\\
            Ikke & \texttt{not} & $!$\\
            enten-eller & \texttt{xor} & $\wedge$\\
          \end{tabular}
        \end{table}
      \end{block}
    \end{column}
    \begin{column}{0.5\textwidth}
\begin{verbatim} 
boolean glad = true;
boolean sur = false;

boolean meh = (glad || sur);
boolean godEksempel = !sur;

\end{verbatim}
    \end{column}
  \end{columns}
\end{frame}



\begin{frame}
  \frametitle{Træning}

  \FIG{codingbat}{Eksempel fra \url{http://codingbat.com/java/Warmup-1}}

\end{frame}




\begin{frame}[fragile]
  \frametitle{Forgreninger}
  \begin{columns}
    \begin{column}{0.5\textwidth}
      \begin{block}{If-else-blokke}
        \texttt{if (\emph{boolean}) \{ }\\
        \texttt{// do stuff} \\
        \texttt{\} else \{ } \\
        \texttt{// do something else} \\
        \texttt{\}}\\
        \vspace{1cm}
      \end{block}
    \end{column}
    \begin{column}{0.5\textwidth}
\begin{verbatim} 
if (x < 200) {
  fill(255, 0, 0);
}

// if-else block
if (x < 200) {
  fill(255, 0, 0);
} else if (x < 300) {
  fill(0, 255, 0);
} else {
  fill(0, 0, 255);
}
\end{verbatim}
    \end{column}
  \end{columns}
\end{frame}


\begin{frame}[fragile]
  \frametitle{Løkker - \emph{while}}
  \begin{columns}
    \begin{column}{0.5\textwidth}
      \begin{block}{looping}
        \texttt{while (\emph{boolean}) \{ }\\
        \texttt{// keep doing stuff} \\
        \texttt{\}}\\
        \vspace{1cm}
      \end{block}
    \end{column}
    \begin{column}{0.5\textwidth}
\begin{verbatim} 
int x = 0;
while (x < width) {
  point(x, 100);
  x++;
}
\end{verbatim}
    \end{column}
  \end{columns}
\end{frame}


\begin{frame}[fragile]
  \frametitle{Løkker - \emph{for}}
  \begin{columns}
    \begin{column}{0.45\textwidth}
      \begin{block}{for loop}
        Ligesom \emph{while} løkke, men tit vil vi gerne bruge en tæller så\\
        \begin{description}
        \item[start] initialisering
        \item[betingelse] Hvor længe skal vi blive ved
        \item[pr-gang] gør noget for hvert gennemløb
        \end{description}
        
      \end{block}
    \end{column}
    \begin{column}{0.55\textwidth}
\begin{verbatim} 
//( start; betingelse; pr-gang)
for(int x = 0; x < width; x++){
  point(x, 100);
}
\end{verbatim}
    \end{column}
  \end{columns}
\end{frame}

\begin{frame}[fragile]
  \frametitle{Mission bold}

  \begin{columns}
    \begin{column}{0.45\textwidth}
      \begin{block}{Mission bold}
        \begin{itemize}
        \item Gem retning i sin egen variabel
        \item Opdater position baseret på retning
        \item Skift retning når bolden rammer kanten
        \end{itemize}
      \end{block}
    \end{column}
    \begin{column}{0.55\textwidth}
\begin{verbatim} 

float boldX = 200;
float boldY = 200;

float deltaX = 2.3;
float deltaY = 1.3;

if ( boldX > width) {
   deltaX = -deltaX;
}

\end{verbatim}
    \end{column}
  \end{columns}
\end{frame}


\begin{frame}
  \frametitle{Mission bold }

  \begin{block}{The mission}
    \begin{itemize}
    \item Lav noget sjovt i draw-metoden
    \item Tilføj alle kanter ( $x < 0$, $Y>height$ \ldots)
    \item Hold pause
    \end{itemize}
  \end{block}
\end{frame}


\begin{frame}[plain]
  % \frametitle{Mere interessant bevægelser}

  \FIGMED{curve}{$s =  \int_{t_1}^{t_2} ds =\int_{t_1}^{t_2} \sqrt{dx^2 + dy^2 + dz^2} = \int_{t_1}^{t_2} \sqrt{\left(\frac{dx}{dt}\right)^2 + \left(\frac{dy}{dt}\right)^2 + \left(\frac{dz}{dt}\right)^2}\; dt$}
  
\end{frame}

\begin{frame}
  \frametitle{Mere interessante bevægelser}

  \begin{block}{Kræfter}
    I virkligheden bliver \emph{partikler} udsat for flere forskellige kræfter.
    \begin{itemize}
    \item Tyngdekraft
    \item Vind
    \item Gnidningsmodstand
    \item Magnetisme\\
      \hspace{.3cm}       \vdots
    \item Fjerderkræfter

    \end{itemize}
  \end{block}

\end{frame}


\begin{frame}
  \frametitle{Vi \emph{snyder}}


  \begin{block}{Aaargh matematik og formler}
    På en eller anden facon er ting der opfører sig naturligt pæne. Fysik og matematik forsøger at beskrive naturen \ldots
  \end{block}


  \begin{block}{Computergrafikkens 1. lov}
    Hvis det ser rigtigt ud, er det rigtigt.
  \end{block}



  Man kan udregne det udfra fysiske love og opstille ganske
  komplicerede ligningsystemer. \\
  Man kan også lave noget det ligner ret godt.

  

\end{frame}

\begin{frame}[fragile]
  \frametitle{Gnidningsmodstand}

  \begin{columns}
    \begin{column}{0.5\textwidth}
      \begin{block}{Bolden}
        delta er hastighed. \\

        Vi ændre hastigheden for at efterligne kræfter.
      \end{block}


      \begin{block}{Gnidnings modstand}
        Vi gør hastigheden mindre og mindre pr frame(tidsskridt).\\
        Ganger delta med et tal tæt på een.
      \end{block}

    \end{column}
    \begin{column}{0.5\textwidth}
\begin{verbatim} 
  float modstand = 0.996;
  deltaX *= modstand;
  deltaY *= modstand;

\end{verbatim}
    \end{column}
  \end{columns}
\end{frame}

\begin{frame}[fragile]
  \frametitle{Tyngdekraft}
  
  \begin{columns}
    \begin{column}{0.5\textwidth}
      
      \begin{block}{Tyngdekraft}
        Påvirker hastigheden så ting falder ned, det vil sige deltaY bliver større.
      \end{block}
      
    \end{column}
    \begin{column}{0.5\textwidth}
\begin{verbatim} 

deltaY += .1;

\end{verbatim}
    \end{column}
  \end{columns}
\end{frame}


\begin{frame}[fragile]
  \frametitle{Prøv det}
  Start bold og tilføj vindmodstand og tyngdekraft
  \begin{columns}
    \begin{column}{0.5\textwidth}
\begin{verbatim}
float boldX = width/2;
float boldY = 2.0 * height/3;
float deltaX = 2;
float deltaY = -2.5;
float modstand = 0.996;
float tyngdekraft = 0.1;
\end{verbatim}
    \end{column}
    \begin{column}{0.5\textwidth}
\begin{verbatim} 
void draw() {
  deltaY += tyngdekraft;
  deltaX *= modstand;
  deltaY *= modstand;
  boldX += deltaX;
  boldY += deltaY;
  //Check grænser
  //Tegn bold
}
\end{verbatim}

    \end{column}
  \end{columns}
\end{frame}


\begin{frame}[fragile]
  \frametitle{\texttt{Tech Tip:}OpenGL}

  At bruge OpenGL til at rendere betyder at arbejdet med at tegne sker
  på grafik kortet.

\begin{verbatim}
import processing.opengl.*;

void setup() {
  size(400,400, OPENGL);
  //mere init
}
\end{verbatim}
\end{frame}

\begin{frame}[fragile]
  \frametitle{Kanon}
  
  \begin{columns}
    \begin{column}{0.5\textwidth}
      
      \begin{block}{Tegne}
        En simpel kanon er et løb, altså en rektangel. \\
        Men løbet skal jo drejes?
      \end{block}

      \begin{block}{Transformationer}
        \texttt{tranlate} og \texttt{rotate} skal angives rigtig
        rækkefølge, og fortæller beskriver hvordan og hvorledes de
        efterfølgende ting tegnes.
      \end{block}

      
    \end{column}
    \begin{column}{0.5\textwidth}
\begin{verbatim} 
//husk radianer!
float cannonA = PI/4.0; 
// tegn kanonen
translate(0, 400);
rotate(cannonA); 
rect(0, -10, 50, 20);

\end{verbatim}
    \end{column}
  \end{columns}
\end{frame}


\begin{frame}[fragile]
  \frametitle{\texttt{Tech Tip:}Radianer}

  Vinkler angives i intervallet $[0 - 2\pi]$, den omvendte y-akse gør at vinklerne kører med uret, modsat normal trigonometri.
  \begin{description}
  \item[\texttt{radians(A)}] Fra grader til radianer
  \item[\texttt{degrees(A)}] Fra radianer til grader
  \end{description}
  \begin{columns}
    \begin{column}{0.5\textwidth}
      \FIGMED{unit_circle}{\url{http://processing.org/learning/trig/}}
    \end{column}
    \begin{column}{0.5\textwidth}
      \FIGMED{unitCircleDegrees.png}{\url{http://btk.tillnagel.com/tutorials/rotation-translation-matrix.html}}
    \end{column}
  \end{columns}
\end{frame}

\begin{frame}[fragile]
  \frametitle{Keyboard input}
  \begin{columns}
    \begin{column}{0.5\textwidth}
      \begin{block}{\texttt{void keyPressed()}}
        Funktionen bliver kaldt, når en tast bliver tastet.\\

        Processing har indbyggede variable:
        \begin{description}
        \item[key] \texttt{char} der beskriver tasten
        \item[keyCode] indbygget variable som : \texttt{LEFT, RIGHT, UP} etc
        \end{description}
      \end{block}
    \end{column}
    \begin{column}{0.5\textwidth}
\begin{verbatim} 
void keyPressed() {
  if (keyCode == LEFT) {
    // drej kanon mod uret
  } else if (keyCode == RIGHT) {
    // drej kanon med uret
    cannonA += .05;
  } else if (key == ' ') {
    // sæt start position
    // sæt start hastighed
  } 
}
\end{verbatim}
    \end{column}
  \end{columns}
\end{frame}

\begin{frame}[fragile]
  \frametitle{Kollisions detektion}
  \begin{block}{ramte jeg noget?}
    Vi er heldige at vore målskive er rund.\\
    d = afstand mellem kugle og målskive. \\
    hvis $d < radius_{skive}+radius_{kugle}$ har vi ramt! 
  \end{block}
  \begin{block}{Afstand}
    processing funktion \texttt{dist(x1, y1, x2, y2)} \\
    ellers $d = \sqrt{ (x_1-x_2) (x_1-x_2) + (y_1-y_2)(y_1-y_2)}$
  \end{block}

\begin{verbatim} 
float d = dist(kugleX, kugleY, targetX, targetY);
if (d < 30) {
   //vi ramte!!
}  
\end{verbatim}

\end{frame}




\begin{frame}
  \frametitle{Udfordringer}

  Ændr \texttt{modstand} og \texttt{tyngdekraft}

  \begin{columns}
    \begin{column}{0.5\textwidth}
      \begin{itemize}
      \item alpha blending: lad vær med at tegne baggrund og brug alpha
      \item tegn en streg fra sidste position til nuværende
      \item tegn anderledes kugle
      \item lad kuglens form og farve afhænge af dens \emph{alder}
      \end{itemize}
    \end{column}
    \begin{column}{0.5\textwidth}
      \begin{itemize}
      \item lad musen tiltrække bolden
      \item multiplayer: flere kanoner + skyd hinaden
      \item blæsevejr
      \item tæl points
      \item ram noget andet
      \end{itemize}
    \end{column}
    
  \end{columns}

\end{frame}



\begin{frame}
  \frametitle{Tak for i dag}

  \begin{itemize}
  \item Hvad syntes \emph{du} om i dag?
  \item Næste gang: flere bolde
  \item $T^3$ i må meget gerne hjælpe med at rydde lokalet.
  \end{itemize}

  \begin{block}{Klasseværelset}
    \url{www.openprocessing.org/classrooms/?classroomID=1075}
  \end{block}

\end{frame}
\end{document}

